%%%% Load Packages %%%%
\usepackage[utf8]{inputenc}
\usepackage[english]{babel}
\usepackage{indentfirst}
\usepackage{color}
% \usepackage{url}
\usepackage{hyperref}
% \usepackage[citestyle=authoryear, bibstyle=authoryear, sorting=nyt]{biblatex}
\usepackage[backend=biber, style=authoryear-asa]{biblatex}
\usepackage{caption}
% \usepackage{etoolbox}
\usepackage{fancyhdr}
\usepackage{geometry} %[margin=1in]
%\usepackage[absolute]{textpos}
\geometry{
	top=1in, % Top margin
	bottom=1in, % Bottom margin
	left=1in, % Left margin
	right=1in, % Right margin
%	showframe, % Uncomment to show how the type block is set on the page
}
\usepackage{pdflscape}
\usepackage{graphicx}
\usepackage{parskip}
\usepackage{setspace}
\usepackage{titlesec}
\usepackage{titletoc}
\usepackage{booktabs}
\usepackage{multirow}
%\usepackage{endnotes} % Use end notes instead of footnotes
%\let\footnote=\endnote % Use end notes instead of footnotes

% Adjust marginparwidth before loading todonotes/changes
\setlength{\marginparwidth}{2cm}
\usepackage{changes} % Use [final] to accept all changes; remove [final] to track changes
% Define author for suggestions
\definechangesauthor[name={Matt Carson}, color=red]{MC}

\usepackage[autostyle, english = american]{csquotes}
\MakeOuterQuote{"}
\usepackage[acronym, nogroupskip]{glossaries}
\makeglossaries

%\titleformat{\section}{\fontsize{12}{14}\bfseries\centering}{\thesection}{0.5em}{}

%%%% Include acronyms file %%%%
%%%% List of Acronyms and abbreviations %%%%

\newacronym{afl}{AFL}{American Federation of Labor}
\newacronym{aflcio}{AFL-CIO}{American Federation of Labor-Congress of Industrial Organizations}
\newacronym{anga}{ANGA}{America's Natural Gas Alliance}
\newacronym{btc}{BTC}{Building Trades Council}
\newacronym{cba}{CBA}{Collective Bargaining Agreement}
\newacronym{cio}{CIO}{Congress of Industrial Organizations}
\newacronym{csb}{CSB}{U.S. Chemical Safety and Hazard Investigation Board}
\newacronym[
    description={Community Workforce Agreement (also called Project Labor Agreement [PLA])}
]{cwa}{CWA}{Community Workforce Agreement}
\newacronym{dapl}{DAPL}{Dakota Access Pipeline}
\newacronym{efca}{EFCA}{Employee Free Choice Act}
\newacronym{iam}{IAM}{International Association of Machinists and Aerospace Workers}
\newacronym{ibew}{IBEW}{International Brotherhood of Electrical Workers}
\newacronym{ilgwu}{ILGWU}{International Ladies Garment Workers Union}
\newacronym{ilwu}{ILWU}{International Longshore and Warehouse Union}
\newacronym{iupat}{IUPAT}{International Union of Painters and Allied Trades}
\newacronym{iww}{IWW}{Industrial Workers of the World}
\newacronym{jatc}{JATC}{Joint Apprenticeship Training Committee}
\newacronym{lcca}{LCCA}{Labor Coalition for Community Action}
\newacronym{lcsp}{LCSP}{Labor Campaign for Single Payer}
\newacronym{liuna}{LIUNA}{Laborers' International Union of North America}
\newacronym{lmrda}{LMRDA}{Landrum-Griffin Labor-Management Reporting and Disclosure Act}
\newacronym{nabtu}{NABTU}{North America's Building Trades Unions}
\newacronym{nepa}{NEPA}{National Environmental Policy Act}
\newacronym{nlra}{NLRA}{National Labor Relations Act}
\newacronym{nlrb}{NLRB}{National Labor Relations Board}
\newacronym{ocaw}{OCAW}{Oil, Chemical and Atomic Workers Union}
\newacronym{opcima}{OPCIMA}{Operative Plasterers' and Cement Masons' International Association}
\newacronym{oshact}{OSH Act}{Occupational Safety and Health Act}
\newacronym{owiu}{OWIU}{Oil Workers International Union}
\newacronym{pace}{PACE}{Paper, Allied-Industrial, Chemical, and Energy Workers International Union}
\newacronym[
    description={Project Labor Agreement (also called Community Workforce Agreement [CWA])}
]{pla}{PLA}{Project Labor Agreement}
\newacronym{proact}{PRO Act}{Protecting the Right to Organize Act}
\newacronym{rwdsu}{RWDSU}{Retail, Wholesale, and Department Store Union}
\newacronym{seiu}{SEIU}{Service Employees International Union}
\newacronym[
	description={United Association of Journeymen and Apprentices of the Plumbing and Pipe Fitting Industry}]
	{ua}{UA}{United Association of Plumbers and Pipefitters}
\newacronym{uaw}{UAW}{United Auto Workers}
\newacronym{ufcw}{UFCW}{United Food and Commercial Workers}
\newacronym{ugccwa}{UGCCWA}{United Gas, Coke, and Chemical Workers of America}
\newacronym{umwa}{UMWA}{United Mine Workers of America}
\newacronym{upiu}{UPIU}{United Paperworkers' International Union}
\newacronym[
	description={United Steel, Paper and Forestry, Rubber, Manufacturing, Energy, Allied Industrial and Service Workers International Union}
]{usw}{USW}{United Steelworkers}

%%%% Head height %%%%
\setlength{\headheight}{15pt}

%%%% Line spacing %%%%
\setstretch{2}

%%%% Paragraph spacing %%%%
\setlength{\parskip}{0pt}

%%%% Define indentation length %%%%
\newlength{\myindent}
\setlength{\myindent}{0.5in}

%%%% Set the hanging indent %%%%
\setlength{\bibhang}{\myindent}

\AtEveryBibitem{
	\clearfield{issn}
	\ifentrytype{book}{\clearfield{isbn}}{}
    \clearfield{urlyear}
    \clearfield{urlmonth}
    \clearfield{urlday}
}

%%%% Redefine the citation command to use a colon instead of a comma and pp. %%%%
\DeclareFieldFormat{postnote}{#1}
\DeclareFieldFormat{multipostnote}{#1}

% Redefine the URL format to not use \texttt style
%\DeclareFieldFormat{url}{\href{#1}{#1}} % This will format the URL in the regular text font

% Optionally, redefine the DOI and eprint formats similarly if you use them
%\DeclareFieldFormat{doi}{\href{https://doi.org/#1}{#1}}
%\DeclareFieldFormat{eprint}{\href{https://arxiv.org/abs/#1}{#1}}

%%%% Use colon (ASA Style) in-text citations (author year: page) %%%%
\renewcommand*{\postnotedelim}{\addcolon}

%%%% Set global text alignment to ragged-right %%%%
%\raggedright

%%%% Paragraph indentation %%%%
\setlength{\parindent}{\myindent}

%%%% Font %%%%
% \setmainfont{Times New Roman}

%%%% Define variables for reused text or dimensions %%%%
% Full title
\newcommand{\fullTitle}{Construction Union Agreements:\par{}Union Organizing in Historical-Comparative Perspective}
% Short title
\newcommand{\shortTitle}{CONSTRUCTION UNION AGREEMENTS}

% Standardize image width
\newcommand{\imageWidth}{0.8\textwidth}

%%%% Page style %%%%
\pagestyle{fancy}
\fancyhf{} % clear all header and footer fields
%\fancyhead[R]{\thepage} % page number on the right side
\fancyhead[R]{\hyperlink{toc}{\thepage}} % page number on the right side, linked to the TOC
\fancyhead[L]{\small \shortTitle}
\renewcommand{\headrulewidth}{1pt} % header rule

% Customize abstract page
\renewenvironment{abstract}
  {\par\noindent\centering\textbf{\fullTitle}\par}
  {\noindent\raggedright}
%  {\par}

% Redefine the quote environment
\renewenvironment{quote}
  {\list{}{\leftmargin=\parindent\rightmargin=0pt}%
   \item\relax}
  {\endlist}
  
% Redefine the quote environment to make it single-spaced 
% and remove vertical space before and add one at the end
\AtBeginEnvironment{quote}{\singlespacing\setlength{\topsep}{0pt}\setlength{\partopsep}{0pt}}
\AtEndEnvironment{quote}{\vspace{0.5\baselineskip}}

% Custom command \acrparen to place acronyms in parentheses
% \acrparen
\newcommand{\acrparen}[1]{(\acrshort{#1})}

% Custom command to reference section number and name
% \fullref
\addto\extrasenglish{\def\sectionautorefname{Section}}
\addto\extrasenglish{\def\subsectionautorefname{Section}}
\addto\extrasenglish{\def\subsubsectionautorefname{Section}}
\newcommand{\fullref}[1]{\autoref{#1}: ``\nameref{#1}"}

% Command to use \textcite in the possessive form
\newcommand{\poscite}[1]{\citeauthor{#1}'s (\citeyear{#1})}