
% Stashing things up here for later

%\begin{quote}
%Quote goes here
%\end{quote}



\documentclass[12pt]{article}

% Packages
\usepackage{setspace} % for adjusting line spacing
\usepackage{parskip} % for controlling paragraph spacing
\usepackage{fontspec} % for setting fonts
\usepackage{fancyhdr} % for custom headers and footers
\usepackage{biblatex} % for bibliography management
\usepackage[margin=1in]{geometry} % for setting margins
\usepackage{titlesec} % for customizing section headings
\usepackage{etoolbox} % for adjusting environment parameters
\usepackage{graphicx} % for including graphics
\usepackage{dot2texi}
\usepackage{tikz}
\usepackage{caption} % for customizing captions
%\usepackage{sectsty} % for modifying section headings

% Line spacing
\setstretch{2}

% Paragraph spacing
\setlength{\parskip}{0pt}

% Define indentation length
\newlength{\myindent}
\setlength{\myindent}{3em}

% Set global text alignment to ragged-right
\raggedright

% Paragraph indentation
\setlength{\parindent}{\myindent}

% Font
\setmainfont{Times New Roman}

% Page style
\pagestyle{fancy}
\fancyhf{} % clear all header and footer fields
\fancyhead[R]{\thepage} % page number on the right side
\fancyhead[L]{\small CONSTRUCTION UNIONS: HISTORICAL-COMPARATIVE}
\renewcommand{\headrulewidth}{0pt} % remove header rule

% Disable hyphenation
%\pretolerance=10000
%\tolerance=2000 
%\emergencystretch=10pt
%\hyphenpenalty=10000
%\exhyphenpenalty=10000

% Bibliography setup
% \addbibresource{My Library.bib}
% \ExecuteBibliographyOptions{sorting=none} % unsorted bibliography

% Define custom section headings
%\titleformat{\section}[block]{\MakeUppercase\normalfont\fontsize{12}{14}\selectfont}{\thesection.}{0.5em}{}
\titleformat{\section}[block]{\normalfont\fontsize{12}{14}\selectfont}{\thesection.}{0.5em}{}
\titleformat{\subsection}[block]{\itshape}{\thesubsection.}{0.5em}{}
\titleformat{\subsubsection}[runin]{\normalfont\itshape}{\hspace{\myindent}\thesubsubsection.}{0.5em}{}[.]

% Redefine the quote environment
\renewenvironment{quote}
  {\list{}{\leftmargin=\parindent\rightmargin=0pt}%
   \item\relax}
  {\endlist}

\renewenvironment{abstract}
  {\par\noindent\centering\textbf{Abstract}\par\noindent\raggedright}
  {\par}

% Redefine the quote environment to make it single-spaced
\renewenvironment{quote}
  {\begin{singlespace}\list{}{\leftmargin=\parindent\rightmargin=0pt}%
   \item\relax}
  {\endlist\end{singlespace}}

\begin{document}

\begin{titlepage}
  \thispagestyle{fancy}
  \fancyhead[L]{Running head: CONSTRUCTION UNIONS: HISTORICAL-COMPARATIVE}
  \centering
  \vspace*{1.95in}
  {\LARGE Construction Union Agreements:\par Union Organizing in Historical-Comparative Perspective\par}
  \vspace{1.2in}
  {Matthew A. Carson\par}
  \vspace{0in}
  {University of California, Los Angeles\par}
  \vspace{0.5in}
  {\today\par}
\end{titlepage}

% Set page numbering to roman for preliminary pages
\pagenumbering{roman}

% Add abstract page
\begin{abstract}
US Building Trade unions organize their workers differently. Most labor unions compel employers to negotiate, but the Building Trades engage in voluntary negotiations, relying on workers' skill levels rather than strike leverage. This approach correlates with their frequent political deviations from the broader US labor movement, particularly in opposing progressive environmental policies and aligning more closely with the petrochemical industry on environmental issues, and not supporting single-payer healthcare. One view is that unions pursue their members’ interests narrowly, sacrificing broader working-class interests if they feel it is necessary to secure work for their members, and some suggest that the conservative stance of the Building Trades stems from their craft union tradition, in which workers are organized by craft and skill instead of by industry. However, using historical-comparative methods, I show that these arguments do not hold. Petrochemical unions have supported progressive policies, and other craft-based unions have endorsed single-payer healthcare. However, unlike the Building Trades, those unions have never used voluntary agreements. Consequently, they have experienced more conflicts with employers. These findings challenge traditional views and suggest that the Building Trades' conservative negotiation strategies significantly shape their political and policy positions, reinforcing an employer-union dynamic that limits challenging management.
\end{abstract}
\newpage

% Add table of contents
\tableofcontents
\newpage

% Set page numbering to arabic for main content
\pagenumbering{arabic}

\section{INTRODUCTION} \

Construction is one of the largest unionized sectors in the United States. In spite of this, labor research has largely not focused on construction unions (or what are often referred to as the Building Trades) but has largely focused on the ways in which workers in other industries organize. In some sense this is understandable. One of the largest periods of union growth in the US was in the 1930s, a period in which multiple violent strikes occurred, sometimes lasting for a month or longer. These events contributed to the rise of the Congress of Industrial Unionism (CIO), which was a federation of trade unions committed to industrial unionism, rather than craft unionism. Congress and President Franklin D. Roosevelt responded to these violent battles by enacting the Wager Act, which guaranteed the right for workers to organize and provided a framework for workers to petition for union representation in the workplace. This framework was largely adapted to the way that industrial unions were organizing at the time. Most unionization efforts, especially large-scale and prominent efforts, have used the industrial union model of organizing.

Construction unions in the US have a distinct approach to organizing their workers vis-à-vis other labor unions. While most labor unions typically compel employers to negotiate either through secret ballot elections or work stoppages, the Building Trades take a different route by engaging in voluntary negotiations. Their strategy hinges more on the skill levels of their workers than the leverage of strikes or official National Labor Relations Board elections, which use the state to compel the employer to negotiate. This unique approach often leads them to deviate politically from the broader US labor movement. Notably, they often oppose progressive environmental policies and tend to align more closely with the petrochemical industry on environmental issues. Additionally, they are not supportive of single-payer healthcare.

Some argue that this conservative stance of the Building Trades originates from their tradition of craft unionism, where workers are organized based on craft and skill rather than industry. However, historical-comparative analyses challenge this view. For instance, other unions with many members working in the petrochemical industry have backed progressive policies, including environmental policies, and other craft-based unions have endorsed single-payer healthcare despite organizing along craft lines instead of industry. The key distinction between these unions and their disparate political stances lies in the Building Trades' use of voluntary agreements, which minimizes conflicts with employers and constrains their ability to challenge management.

% Foner, Phillip S. History of the Labor Movement in the United States: Volume 2: From the Founding of the AFL to the Emergence of American Imperialism. New York: International Publishers, 1955; pp. 132–133.

\subsection{Craft Unionism vs. Industrial Unionism} \

Craft unionism is when workers are organized into a union by craft (occupation) rather than employer or industry. The earliest efforts in the US to organize workers collectively to assert the interests and preferences were generally organized around craft lines. Plumbers formed their union; machinists started one; so did the carpenters, and so on. Many of these early craft unions were successful, and eventually The American Federation of Labor (AFL) formed in 1886, bringing many of the labor unions at the time into a nationwide federation. Immediately many the constituent unions battled over craft jurisdiction. Who had the right to organize workers in the print room? Both the Machinists and the International Typographical Union claimed jurisdiction. As a result, one of the most important functions of the newly formed federation was to quell these jurisdictional fights.

Industrial unionism did not come until some time later. Industrial unionism is when workers are organized by employer or industry, regardless of their craft, occupation, or skill level. Many of the early craft unionists in the AFL were skeptical of indsutrial unionism. Many of the workers in mass production factories at the time were immigrants with no craft and little formal training. The craft union leaders in the AFL did not think that it would be possible to organize them because they did not have leverage that having a craft would equip the workers with. The jobs were largely repetitive and did not require a high-skill level. For example, standing in the same place and fastening the same bolts at the same place on a Model-T all day was not a craft in the way that being a machinist and working a lathe or being a pile driver and driving piles for a new bridge was.

\subsubsection{The Industrial Bargaining Approach}

The National Labor Relations Act (NLRA) sets forth the process that workers must follow to unionize a workplace. The National Labor Relations Board (NLRB) administers the NLRA. The NLRA specifies the processes that workers and union organizers should follow to form a union. The typical process begins with an effort to determine if there is a general interest in forming a union or joining a union among workers already hired. If there appears to be enough interest, “employee organizers typically collect union interest cards, petitions or other written statements from co-workers to show interest in union representation. Organizing efforts may be supported by an established union seeking to represent workers at a workplace. Workers may also form an independent union” (DOL n.d.:How can I form a union?). If enough cards are collected to demonstrate majority status, the workers may ask the employer to voluntarily recognize the union. If the employer refuses to voluntarily recognize the union, workers may strike and/or file a petition with the NLRB to request an election to certify the union and the collective bargaining representative of the workers (DOL n.d.; NLRB n.d.). Once the union has demonstrated that the majority of workers want union representation (either through voluntary recognition by the employer or through a NLRB election), the union and employer will attempt to reach an initial agreement. This type of agreement is called a Section 9(a) agreement.

\subsubsection{The Building Trades Approach}

In contrast to this approach, building and construction trade unions do not usually organize the workers at the workplace. They usually establish voluntary agreements with employers, which can be established before any employees are hired by the employer. Since these agreements can be established before any workers are hired, they are called prehire agreements, or in legalistic jargon, Section 8(f) agreements. Prehire agreements have a long history that predates their legitimation by Congress in 1959 with the passage of the Labor-Management Reporting and Disclosure Act (LMRA). In fact, prior to the LMRA, the NLRB had refused to take jurisdiction over the building and construction trades because their approach to union organizing was so different and in fact would be an unfair labor practice, a violation of the NLRA, if they were to take jurisdiction over the matter. So, whereas labor law is often discussed in terms of how the law constrains union organizing efforts, this is not the case with respect to prehire agreements used by the building and construction trades; they were already organizing this way prior to the passage of the LMRA.

This distinction between 9(a) and 8(f) agreements is useful because it underscores two very different orientations toward the boss and two very different approaches to mobilizing worker power. Perhaps most notable is the starting point of the unionization process. The former starts with the employees already hired. Sometimes it is primarily driven by union staff; other times, it can be driven more by shop floor organizers from the rank-and-file. Either way, the approach is to organize workers around issues they are facing in the workplace and compel the employer to negotiate a contract through coercive measures such as strikes or using the state as an instrument to compel the employer to bargain in good faith. The building and construction trade unions largely dispense with this approach, instead trying to entice employers on the basis of their members’ skill and training and by collaborating with the employers. (The building and construction trades are not forced by law to organize in this way; it is their choice whether to organize in this way or another way.)

\begin{figure}
  \centering
  \includegraphics[width=0.75\textwidth]{images/organizing_paths}
  \captionsetup{justification=centering, singlelinecheck=false, margin=2cm}
  \caption{The industrial mode of organizing (left), and the construction mode of organizing (right). Construction unions may follow either path, but other unions may not voluntarily negotiate the way that construction unions can.}
  \label{fig:organizing_paths}
\end{figure}

One practical reason for the development of prehire agreements in the building and construction trades is the temporary nature of the work. Because of the temporary nature of the work, many union halls maintain a hiring hall with dispatch procedures for members out of work. At the completion of the job assignment, laid off union members may return to the union’s hiring hall and make themselves available for employment somewhere else by signing the out of work list maintained by the union. The union will dispatch its members to jobs as needed by the employer. Because of this temporary nature of the job, organizing via the traditional route is much more challenging because workers are laid off so frequently and the employer has no permanent workforce, at least not in the way that other workplaces do. From that perspective, prehire agreements are a practical response to the specific unionization challenges faced in the construction industry. However, this particular arrangement has serious consequences for how construction unions must situate themselves in relation to the employer. That is to say, because of the voluntary nature of the agreements, the employer may simply choose to leave the collective bargaining relationship at the expiration of the contract. Thus, construction unions that follow this method of organizing generally must maintain a much more employer-friendly orientation than other unions who organize through the other route. Simply put, if the employer feels that the workers are pushing too hard in their enforcement of their contract or that the union has become too radical, the employer is free to leave at the next bargaining round.

Because of this threat, building trades unions have to be employer friendly. For example, the union does not have exclusive control over its training program. Instead, the Joint Apprenticeship Training Committee (JATC) oversees the training process and curriculum development. The JATC is a committee comprised of 50% labor and 50% management, each having equal share in controlling the apprenticeship program. 

\section{THE HIRING HALL}

The temporary nature of construction work gave rise to a different mode of organizing. Because of the temporary nature of construction work, construction unions (of the American Federation of Labor, AFL) organized a much different set of arrangements than the industrial unions (mostly in the Congress of Industrial Organizations, CIO) that came along later to organize around industry-wide lines. Rather than organize the workers on the shop floor to demand concessions and better working conditions from their employer, construction unions use their leverage as a particularly skilled workforce that employers needed to complete their projects. Because of this employer need for a particular skill set, the union workers could demand wages, working-conditions, fringe benefits, etc. In exchange for their labor. If employers are not willing to cede these demands, the skilled workers can withhold their labor, and employers will not be able to complete their projects. If employers are able to reach an agreement with the union, then a contract is crafted and a hiring hall established (or the employer is incorporated into an extant hiring hall). The hiring hall is a central component of the prehire contract arrangement. They go hand-in-hand. Its development is largely due to the particularly contingent nature of construction work:

\begin{quote}
	Many construction industry employers hire employees, as the need arises, to work on a particular project and to be laid off when their services are no longer required. Most construction workers are organized into union hiring halls. A hiring hall, or work referral system, is an arrangement under which a union that "has control of or access to a particular labor pool agrees to supply workers to an employer upon request." When an employer in the construction industry needs skilled workers for a project, he often seeks such workers from the union hiring hall. Historically, the practice in the construction industry was for the employer to sign an agreement with the hiring hall union which set the terms and conditions of employment for workers \textit{not yet hired}. These contracts, known as pre-hire agreements, often contemplated a tenure of years, spanning several projects. Rather than having to renegotiate the terms and conditions of employment on each new project, the employer was assured a ready supply of skilled workers and predictable labor costs upon which to base his bids on projects subcontracted by a general contractor. Furthermore, the construction worker had the benefit of a central clearinghouse for employment opportunities. (emphasis added; Murphy 1982 pp. 1014–1015)
\end{quote}

Construction unions had been operating in this way for many years prior to the exemption to majority status requirements that were carved out for construction unions in Section 8(f) as part of the Landrum-Griffin Labor-Management Reporting and Disclosure Act (LMRDA) of 1959. In fact, before its enactment, the prehire agreement constituted a violation of the law which had, up until that point, required the union to show that the majority of employees wanted representation from that particular union. Murphy (1982) explains how the particularities of the construction industry made other modes of union organizing and operation difficult:

\begin{quote}
	The periods of employment in the construction industry are often so short, however, that it is impracticable to complete the process of certifying a collective bargaining representative before a project ends and the employees are laid off. Furthermore, if an employer recognizes a union as bargaining representative of his employees, but the union is not in fact supported by a majority of his employees, the employer has committed the unfair labor practice of illegally assisting a union in violation of section 8(a)(1) of the National Labor Relations Act (NLRA). Pre-hire agreements, therefore, technically constituted illegal assistance of a union by the employer because agreements were signed prior to the union attaining majority status, indeed before the employer had even hired any employees. (pp. 1016-1017)
\end{quote}

To which the NLRB’s response was to ignore the violation until the 1959 passage of the LMRDA:

\begin{quote}
	The General Counsel for the Board recognized the need of the construction industry to have a skilled work force available for quick referral and adopted a policy of not issuing complaints against construction employers and unions entering into pre-hire agreements. Finally in 1959, Congress legitimized pre-hire agreements by enacting section 8(f) of the NLRA. (Murphy p. 1017)
\end{quote}

In Senate debate, many senators acknowledged that this way of organizing in the construction trades was long-standing and informally ignored by the Board and courts. For example, Senator Javits recognized that the Taft-Hartley law, which LMRDA was to amend, was not being applied at the time:

\begin{quote}
	We cannot apply the Taft-Hartley law to the building and construction field. We all know the law is not being applied in that field, and we might as well recognize the fact in the law. This is an essential amendment, and the sooner we adopt it the better. (“105 Congressional Record, 86th Congress, 1st Session (1959)”, p. 6395) 
\end{quote}

Nor was this a regulation "imposed from above" to stifle trade unionism the way that Taft-Hartley was a dozen years earlier. Senator Dirksen acknowledged that labor wanted this amendment and had even approached senators about it in the years before LMRDA’s passage:

\begin{quote}
	I remember that 2 years ago, when Mr. Richard Gray, the head of the Carpenters Union, came to the reception room with the General Counsel of the Department of Labor, they presented to us a proposal on the so-called prehiring agreements in the construction industry. They said, "This is what we want." (105 Congressional Record, 86th Congress, 1st Session (1959), p. 6414)
\end{quote}

A few years earlier in 1956, the United Brotherhood of Carpenters’ magazine, the Carpenter, the official publication of the union spoke favorably of the bill because it would allow contractors and building trades unions [to] carry on contractual relationships in the way they have for over 50 years" (https://archive.org/details/carpenter76unit/page/n5/mode/2up?view=theater pp. 7-8)

\begin{figure}
  \centering
  \includegraphics[width=0.8\textwidth]{images/hiring_hall}
  \captionsetup{justification=centering, singlelinecheck=false, margin=2cm} 
  \caption{Construction unions usually maintain a hiring hall. When employers require more workers for their projects, they contact the hiring hall to request workers to be dispatched to the job site.}
  \label{fig:hiring_hall}
\end{figure}

\begin{figure}
  \centering
  \includegraphics[width=0.8\textwidth]{images/union_power_red}
  \captionsetup{justification=centering, singlelinecheck=false, margin=2cm} 
  \caption{Since negotitations between construction unions and employers are voluntary, construction unions typically have more leverage where the employer requires a more skilled workforce or where the job is large and requires many employees.}
  \label{fig:union_power_red}
\end{figure}

\section{THE CASES} \

Cases that shared similar features with the building and construction trades were selected for comparison. The OCAW/USW, a union that has many members working in the petrochemical industry, and the Machinists, a craft union that does not organize workers the way that the building and construction unions do. Lastly, a intra-building trades union comparison is made between the general building trades orientation and those building trades unions who have chosen to conduct NLRB elections in the same way that non-construction unions do.

\subsection{The United Association of Plumbers and Pipe Fitters (UA) Local 189} \

In his masterful study of the United Association of Plumbers and Pipe Fitters Local 189 in Colombus, Ohio, Richard Schneirov details the uniqueness of construction unionism. The formation of the United Association of Plumbers and Pipe Fitters began in the late 1880s, Before then, plumbers and pipe fitters were largely unorganized; to the extent that they did form unions, the organizations were usually temporary, to meet a specific need at a particular time and were hyper-local and fragmented, without any international (a term that, in the context of US and Canadian unions, means a national organization) to push for the interests of plumbers and pipe fitters across the nation (p. 58). The first boon for the pipe tradesmen was not an organizing campaign undertaken by the union but the American Federation of Labor’s eight-hour-day movement (pp. 11, 43-45). Many pipe tradesmen in the late 1880s and early 1890s worked upwards of ten hours a day, sometimes up to fourteen hours per day. This movement was “of absolutely crucial importance. Hitherto, they had only been able to mobilize large groups of workers for short periods of time” (pp. 43-45). It also broadened solidarities, appealing to workers of “all natalities, races, trades, industries, skills levels, and genders” (p. 43). Indeed, though the eight-hour-day movement was largely associated with unionism, which provoked strong resistance from employers, it even appealed to non-union workers (p. 45). From that perspective, it was also able to “cement union sentiment and loyalty among large numbers of nonunion workers” (p. 45). And in contradistinction to earlier union efforts, this brought unions together across the nation, culminating in a nation-wide eight-hour-day strike on May 1, 1886. However, three days later the movement suffered a major setback when anarchists bombed the Haymarket Square in Chicago, killing seven police officers (p. 45).

However, the eight-hour movement’s efforts proved to be more durable, and on November 15, 1889, it gave rise to the formation of the first union of journeymen plumbers and pipe fitters Local 5180 in Columbus (p. 45). In 1890, Columbus streetcar workers continued to push for shorter workdays. They demanded a reduction from ten to nine work hours per day. In this context, the plumbers and pipe fitters were well positioned to exact concessions from the master plumbers because of fears that the latter could not defeat the former (pp. 45-46). The master plumbers agreed to a reduction in work hours without a loss in pay. However, the Local was inexperienced in negotiations with the master plumbers, and they were willing to make their own concessions to the employers that would allow the latter to severely restrict the time when overtime pay would start (pp. 46-47). The union sought the advice of the AFL founder Samuel Gompers, who disabused the local union from making concessions (p. 46-47). The two sides eventually reached a written agreement, the first of its kind in the trade. 

This encouraged the formation of the city’s first Building Trades Council (BTC). Despite earlier challenges in uniting the trades, the Council brought all crafts together: bricklayers, lathers, painters, carpenters, tinners, and plumbers and pipe fitters, under the banner popularized by the Knights of Labor: “An injury to one is the concern of all” (p. 47). And notwithstanding the master plumber’s earlier concessions, this inter-union trade solidarity set the stage for even more militancy. The BTC opted, in the words of the Columbus Dispatch, “for ‘radical measures’ at the opening of the ensuing building season” (p. 47). They demanded that union workers, no matter the craft or trade, perform the work on every union job site (p. 47). If their demand was not honored, the BTC threatened to not “touch any job” where non-union workers were present. Simply put, this meant that if one craft were union on a job site, that union workers on the site would refuse to work there unless every other craft were also unionized.

But by 1892 the union had pivoted to issues concerning the integrity of the craft. The union framed this in the language of public health, as the Columbus Trades and Labor Assembly contended that “nearly all the houses that are built for renting purposes are fitted with bad plumbing causing disease and death” (pp. 47-48). In response to these issues, the Assembly pushed for the city to adopt building codes and endorsed a union plumber for city inspector (p. 48). Consistent with this, the union began to place greater emphasis on the development of an apprenticeship program; the union asked the master plumbers for another raise and the establishment of an apprenticeship program “governing the number and conduct of apprentices” (p.48). The master plumbers refused, and the union went on strike. The master plumbers brought in strikebreakers, and the union eventually lost.

But the plumbers and pipe fitters union had not entirely abandoned broader solidarities. They continued to be involved in the BTC and in the Trades and Labor Assembly in an effort to “advanc[e] labor’s political strength” (p. 50). One leader epitomized the class-based orientation of the era: Louis Bauman. Bauman was the vice president of the Trades and Labor Assembly in 1893, and, beginning in 1894, he also served as the president of Local 57 of the plumbers and pipe fitters. He “was a Labor-Populist, aligned with radical farmers in the Farmers’ Alliances and the national People’s party and with the union men who felt that something should be done to change an economic system in which workingmen were impoverished while Wall Street banks and national corporations dominated the government” (p. 50). The Columbus Trades and Labor Assembly’s 1894 constitution exemplified its radical principles:

\begin{quote}
	It is self-evident that, as the power of capital combines and increases, the political freedom of the masses becomes more and more a delusive force. There can be no harmony between capital and labor under the present industrial system for the simple reason that capital, in its modem character, consists very largely of rent, interest, and profits, extorted from the producers, who possess neither the land nor the means of production, and are therefore compelled to sell their labor and brains or both to the possessor of the land and means of production at such prices as an uncertain and speculative market may allow. Organization of Trades and Labor Unions is one of the most effective means to check the evil outgrowths of the prevailing system. (p. 50)
\end{quote}

However, the radicalism of the nascent pipe trades union was fraught with contradictions. In the early 1890s, the union had moved to exclusive agreements. These agreements offered lucrative benefits to entice the master plumbers into a contract. It not only limited the smaller firms’ ability to compete with the master plumbers by requiring union plumbers and fitters to work exclusively for the master plumbers, but it also precluded workers from “handl[ing] any materials not purchased by their immediate employers” (p. 54). In exchange the master plumbers agreed to allow “the union to set standard rates for the trade,” and “offered local unions benefits that they had great difficulty winning otherwise: a closed shop, the eight-hour day, and stable or increased wages” (p. 54). At the same time, these agreements undercut the union’s bargaining power and leverage and, in many ways, the interests of union members. In particular, it limited employment opportunities for union members because non-signatory contractors could not employ union members even when paying union wages and adhering to union rules; this also hindered “the freedom of action of the unions” (p. 54). Eventually, the contractors’ demands undercut the union’s power so much that the union abandoned the practice of exclusive agreements in 1899 (p. 54-55).

Ironically, the abandonment of these exclusive agreements did not trigger union militancy or radicalism, and by the early 1900s, the union had all but entirely abandoned any commitment to improving working conditions; significantly, they also abandoned the broader labor solidarities that had defined the earlier period. If one word were to characterize this period, it would be “stability.” Many unions in the late 1800s were temporary organizations. Indeed, as Schneirov observed, the telephone book and the UA’s official journal suddenly ceased to list Columbus UA Local 57 in 1898; it had completely disappeared without any historical record of what precipitated its collapse (p. 51). To achieve stability, American craft unions, following in the tradition spearheaded by the British craft unions, rose their union dues and implemented a quasi-welfare state. At the time, there was no “social safety net” provided by government; no unemployment benefits; no workers compensation; no social security benefits; and no Medicare or Medicaid. With increased dues, unions were able to finance these benefits and, consequently, keep construction workers in the union much longer than they had previously (p.59). The Columbus plumbers and pipe fitters union could also afford to hire professional staff and leadership for the first time. The union hired full-time business agents, who served as “walking delegates,” policing the job sites for contract violations and craft jurisdictional violations by the employers (p. 60). He also was an organizer, enrolling new members, and “pulling a job” if he could not resolve a workplace dispute through negotiation (p. 60). And though most employers did not utilize it, in November 1907, the union established a hiring hall as an institution that would later become a staple of construction unionism; the hiring hall required laid-off members to provide the business agent their names so that he could register them on the list of unemployed members and dispatch them, beginning with the first member laid off to the member most recently laid off, at the employers request (pp. 60-61). And finally, this era ushered in intra-craft solidarity: plumbers and fitters for the first time began to see themselves not as individuals trying to “better themselves in the marketplace, hoping to ultimately become independent masters, but as workers with shared interests with other plumbers and fitters (p. 61). These developments made the union into a more durable institution.

\subsubsection{A contemporary view} \

Schneirov contends that the plumbers and pipefitters have had to balance between two sometimes contradictory identities. One is that they are workers with a set of common class interests with other workers in a labor movement built around solidarity for “union brothers and sisters.” But from another perspective, they are part of a craft community that values craftsmanship and that takes pride in its work, a value that they share with their employer (Schneirov 1993:3–4) . A construction journeyworker is distinct in this regard from other blue-collar workers, such as factory workers on an assembly line, who do a repetitive task that can be easily taught (Schneirov 1993:5). Before one can become a journeyworker, one must complete an apprenticeship, which requires learning how to apply general plumbing and pipe fitting installation techniques, principles, and codes to specific circumstances. This requires a rigorous training program to prepare apprentices for their career. Because of this, these workers have gained prestige and a sort-of elite status that many other blue-collar workers could only dream of.

At the same time this elite status and craft pride also comes with a much higher degree of collaboration and much closer relationship with the boss. In fact, in Local 189, as is the case with many other plumber locals across the country, there can be a blurred line between employer and union member, and some union collective bargaining agreements even allow for contractors, that is, the owners of the enterprises, to work on projects alongside union employees (Schneirov 1993:5). This is also the case with UA Local 342 in the San Francisco Bay Area. Local 342 allows employers no more than one owner of the company to “work with the tools,” so long as “ the Individual Employer has not more than two (2) journeymen and one (1) apprentice dispatched” (Local 342 MLA). Many of these employers who also work “work with the tools” are members of the union. For example, Brown 3 Plumbing in Oakland, California is owned by William Brown, a Local 342 member who also works on many of his company’s projects (website). LJ Kruse Company in Berkeley, California similarly has members of the Kruse family who also work on the job site. Will Kruse has both completed the 5-year apprenticeship and serves as the company’s Vice President and Service Manager. He can be seen on the company’s website donning construction gear with a dirty high-visibility vest and jeans on a job site with a pile of steel framing in the background. And at Local 159 in Martinez, California, Brian Lescure, part of the Lescure family that owns Lescure Company, is both the Union’s Apprenticeship Coordinator and elected to the Union’s Examining Board (Lescure n.d.; Local 159 n.d.).

This has blurred line can give workers a sense, much more so than in other industries, that both employer and employee are on the same “team.” And unlike an industry such as manufacturing, where large amounts of capital are necessary to start a business, starting a plumbing business is not unrealizable. One survey even found that half of a large plumbers local “had thought of entering business on their own at one time or another, though most had not done so” (Schneirov 1993:5). Put straightforwardly, this means that half of the local union’s workers, that is, sellers of labor power, also imagine a future as buyers of that same labor power. Schneirov contends that this blurring of the lines and class collaborative dynamic creates a “craft community,” where employers and workers share a common background that “breed an ethic of cooperation among individuals based on mutual respect for craft knowledge, skill, and ingenuity” (1993:6). And “Even those union members who have never considered contracting have often bid independently on small jobs and are familiar with the psychology of being an entrepreneur” (Schneirov 1993:5,6). 



\subsection{The Oil Chemical and Atomic Workers (OCAW) / United Steelworkers (USW)} \

The United Steel Workers (USW) and the Building Trades both have substantial work in the petrochemical industry. The Oil Chemical and Atomic Workers Union (OCAW) represented workers in the petrochemical industry for much of the 20th Century, but by the 1980s, membership numbers began to decline significantly. They merged with several other unions to form the Paper, Allied-Industrial, Chemical and Energy Workers International Union (PACE) in 1999. However, that merger only lasted 6 years, and they merged with the United Steel Workers in 2005.

The Oil Chemical and Atomic Workers were much more radical than other unions, and as Mark Dudzic, a former leader and retired member once described it, “The oil industry never really accepted the union as a junior partner. The union was never able to win the union shop and all the other accoutrements of class peace. As a result, the culture of militancy was deeply embedded in the union” (Leopold 2007). Though the OCAW is now dissolved, and has gone through two mergers, it is worth tracing the history to the present day to analyze if their radical history has had any lasting effect, and if the lack of class peace persists.

Unlike the Building Trades, the USW is currently a supporter of the Labor for Single Payer Health Care campaign, which is a broad social-wage political effort. From that perspective, the USW is not just focusing singularly on their members’ interests but is instead committed to broader political struggle. Strikingly, the USW even supports a transition from dirty-energy to clean-energy jobs. This is a significant thing for a union that has many workers in the petrochemical industry. Such a transition would end their employment at oil refineries, and require a new, more challenging struggle to establish a union foothold in new clean-energy sectors, sectors that have been highly resistant to unionization. Nevertheless, the USW has adopted a broader, more long-term vision, one much more solidaristic with other environmental activist groups than the Building Trades have.

Take the Dakota Access Pipeline as an example. The Building Trades were staunch supporters of the project, even while the rest of the labor movement was largely opposed. The latter aligned themselves with the environmentalists, and the Native Americans at the Standing Rock Reservation, while the former aligned themselves with the American Petroleum Institute and petrochemical employers.

\subsection{The International Association of Machinists (IAM)} \

The International Association of Machinists (IAM) and Building Trades both are unions in the craft union mold. Both were American Federation of Labor unions (before the 1955 merger), which was a labor federation largely rooted in craft unionism. Although perhaps not anchored as steadfastly to the expansion of social-wage policy and decommodified public goods as unions such as the National Nurses United and National Health Care Workers or even The Brotherhood of Maintenance of Way Employees, they have nonetheless given their official support toward the effort to establish single-payer healthcare in the United States, a policy that would eliminate private insurers and establish government-funded healthcare insurance that would cover all regardless of ones income or finances. This arguably is along the more "radical" edge of the US trade union movement, such as it is.

What might cause a union that comes out of the craft union mold typically associated with very conservative politics and historically with racist, exclusionary policies and practices to support progressive social-wage policy that would cover not only its members, but the entire US population regardless of race, gender, ethnicity, or immigration status? One possible response is that it is ideological, that the Machinists have always been guided by a strong commitment to what Joe Burns has termed, "class struggle unionism." The issue with this is explanation is that it does not do much more than repeat the question by way of answer. That is to say, why are there these ideological differences? The other problem, which deserves further investigation, to be sure, is that that contention likely has little to no empirical support.

The alternative hypothesis is that the material relations of production, forged in the context of particular bargining schemes, strategies, which often enough, are responses to pragmatic concerns and constraints, condition the way in which union workers and union leadership adopt a guiding ideology and orientation vis-a-vis the bosses. If the approach is successful, it continues into successive contracts and generations. This might be summed up with the common aphorism, "if it ain’t broke, don’t fix it." While this likely does not follow a strict determinism (for example, political pressures and conditions outside the union and organizers and agitators within the union advancing a different orientation vis-a-vis the boss exert their own forces), it nonetheless exerts force and places constraints on how union workers and leaders relate to the boss.

With respect to the building trades in particular, they are responding to a particular set of historical circumstances that have placed constraints on their relationship with the employers. The most obvious one is the contingent nature of construction work. While in other industries and occupations, employees often work for the same company for their entire career (or at least for a duration spanning many years), construction workers are employed by a single employer anywhere from 1-2 weeks to 1-2 years.  This precariousness makes organizing the typical route (organizing the workers \textit{already employed)} impractical. Although the National Labor Relations Act (NLRA) often permits workers who were employed at the time the election was requested to vote, how effective of a strategy would it be to organize a workplace where it is in essence a revolving door in this fashion? From that perspective, "organizing the bosses" is a logical response to the question. But it also requires a strategy that is largely \textit{anti-conflictual}.

On the other hand, the IAM, which is still anchored in the craft union orientation, does not have this "revolving door" problem. Much of their workforce is permanent, in fields such as manufacturing, on-site installation and service work.  Because of this difference in duration of employment, the IAM does not "organize the bosses" the way that the Building Trades does. This means that they can adopt a more confrontational strategy, one that does not seek to find common ground with the employer but one that compels the employer to respond to their demands and offer concessions. It is worth noting that this approach to organizing does not preclude class collaboration. Many of the unions that follow this route, for example the United Food and Commercial Workers, have notoriously opposed conflictual strategies. However, this approach at the very least leaves the opportunity for conflictual strategy vis-a-vis the boss open as a possibility, while the Building Trades approach largely precludes it because of their voluntary contracts. 

\subsection{Intra-Building Trades Cases} \

While the Building Trades largely organizes the bosses, a wonderfully apt description offered by Mark Dudzic, rather than the workers already employed, there is still some variation in how they organize. A search of the National Labor Relations representation case petition database revealed that approximately 98 building trades unions have filed for a representation election in 2023. The results were obtained by removing company names that did not contain the words “construction,” “constructors,” “electrical,” “plumbing,” “mechanical,” “builder,” or “builders.” If the certified representative matched the names of any of the Building Trades’ constituent unions, they were added to the list as well. While this method did not filter out all non-Building Trades unions, it probably erroneously excluded some from the filtered list because it is nearly impossible to include ever possible search term.

This provides an interesting opportunity because building trades local unions that petition for a representation election are not following the usual organizing path. They are organizing the workers instead of the bosses. Interviewing the union organizers to understand what drove this choice of organizing over the usual building trades approach could produce interesting findings. Perhaps the union had to become creative in a difficult construction economy that threatened to reduce their membership. Or perhaps workers in those shops reached out to the union because of workplace issues or low wages, and they sought unionization to improve their working conditions and pay.

In any event, it might be most useful for the purposes of this project to discover if these local unions are outliers within the Building Trades. That is, do they have a more militant or more conflictual orientation toward the employer? And how do they orient themselves in relation to broader progressive movements, social-wage policies, and environmental groups? 




\newpage

\begin{center}
{\bfseries Notes}
\end{center}

\noindent
.
\newpage
\begin{center}
{\bfseries Bibliography}
\end{center}

% \printbibliography

\end{document}
