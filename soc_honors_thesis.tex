
% Stashing things up here for later

%\begin{quote}
%Quote goes here
%\end{quote}



\documentclass[12pt]{article}

% Packages
\usepackage{setspace} % for adjusting line spacing
\usepackage{parskip} % for controlling paragraph spacing
\usepackage{fontspec} % for setting fonts
\usepackage{fancyhdr} % for custom headers and footers
\usepackage{biblatex} % for bibliography management
\usepackage[margin=1in]{geometry} % for setting margins
\usepackage{titlesec} % for customizing section headings
\usepackage{etoolbox} % for adjusting environment parameters
\usepackage{graphicx} % for including graphics
\usepackage{dot2texi}
\usepackage{tikz}
\usepackage{caption} % for customizing captions
%\usepackage{sectsty} % for modifying section headings

% Line spacing
\setstretch{2}

% Paragraph spacing
\setlength{\parskip}{0pt}

% Define indentation length
\newlength{\myindent}
\setlength{\myindent}{3em}

% Set global text alignment to ragged-right
\raggedright

% Paragraph indentation
\setlength{\parindent}{\myindent}

% Font
\setmainfont{Times New Roman}

% Page style
\pagestyle{fancy}
\fancyhf{} % clear all header and footer fields
\fancyhead[R]{\thepage} % page number on the right side
\fancyhead[L]{\small CONSTRUCTION UNIONS: HISTORICAL-COMPARATIVE}
\renewcommand{\headrulewidth}{0pt} % remove header rule

% Disable hyphenation
%\pretolerance=10000
%\tolerance=2000 
%\emergencystretch=10pt
%\hyphenpenalty=10000
%\exhyphenpenalty=10000

% Bibliography setup
% \addbibresource{My Library.bib}
% \ExecuteBibliographyOptions{sorting=none} % unsorted bibliography

% Define custom section headings
%\titleformat{\section}[block]{\MakeUppercase\normalfont\fontsize{12}{14}\selectfont}{\thesection.}{0.5em}{}
\titleformat{\section}[block]{\normalfont\fontsize{12}{14}\selectfont}{\thesection.}{0.5em}{}
\titleformat{\subsection}[block]{\itshape}{\thesubsection.}{0.5em}{}
\titleformat{\subsubsection}[runin]{\normalfont\itshape}{\hspace{\myindent}\thesubsubsection.}{0.5em}{}[.]

% Redefine the quote environment
\renewenvironment{quote}
  {\list{}{\leftmargin=\parindent\rightmargin=0pt}%
   \item\relax}
  {\endlist}

\renewenvironment{abstract}
  {\par\noindent\centering\textbf{Abstract}\par\noindent\raggedright}
  {\par}


\begin{document}

\begin{titlepage}
  \thispagestyle{fancy}
  \fancyhead[L]{Running head: CONSTRUCTION UNIONS: HISTORICAL-COMPARATIVE}
  \centering
  \vspace*{1.95in}
  {\LARGE Construction Union Agreements:\par Union Organizing in Historical-Comparative Perspective\par}
  \vspace{1.2in}
  {Matthew A. Carson\par}
  \vspace{0in}
  {University of California, Los Angeles\par}
  \vspace{0.5in}
  {\today\par}
\end{titlepage}

% Set page numbering to roman for preliminary pages
\pagenumbering{roman}

% Add abstract page
\begin{abstract}
US Building Trade unions organize their workers differently. Most labor unions compel employers to negotiate, but the Building Trades engage in voluntary negotiations, relying on workers' skill levels rather than strike leverage. This approach correlates with their frequent political deviations from the broader US labor movement, particularly in opposing progressive environmental policies and aligning more closely with the petrochemical industry on environmental issues, and not supporting single-payer healthcare. One view is that unions pursue their members’ interests narrowly, sacrificing broader working-class interests if they feel it is necessary to secure work for their members, and some suggest that the conservative stance of the Building Trades stems from their craft union tradition, in which workers are organized by craft and skill instead of by industry. However, using historical-comparative methods, I show that these arguments do not hold. Petrochemical unions have supported progressive policies, and other craft-based unions have endorsed single-payer healthcare. However, unlike the Building Trades, those unions have never used voluntary agreements. Consequently, they have experienced more conflicts with employers. These findings challenge traditional views and suggest that the Building Trades' conservative negotiation strategies significantly shape their political and policy positions, reinforcing an employer-union dynamic that limits challenging management.
\end{abstract}
\newpage

% Add table of contents
\tableofcontents
\newpage

% Set page numbering to arabic for main content
\pagenumbering{arabic}

\section{INTRODUCTION} \

Construction is one of the largest unionized sectors in the United States. In spite of this, labor research has largely not focused on construction unions (or what are often referred to as the Building Trades) but has largely focused on the ways in which workers in other industries organize. In some sense this is understandable. One of the largest periods of union growth in the US was in the 1930s, a period in which multiple violent strikes occurred, sometimes lasting for a month or longer. These events contributed to the rise of the Congress of Industrial Unionism (CIO), which was a federation of trade unions committed to industrial unionism, rather than craft unionism. Congress and President Franklin D. Roosevelt responded to these violent battles by enacting the Wager Act, which guaranteed the right for workers to organize and provided a framework for workers to petition for union representation in the workplace. This framework was largely adapted to the way that industrial unions were organizing at the time. Most unionization efforts, especially large-scale and prominent efforts, have used the industrial union model of organizing.

Construction unions in the US have a distinct approach to organizing their workers vis-a-vis other labor unions. While most labor unions typically compel employers to negotiate either through secret ballot elections or work stoppages, the Building Trades take a different route by engaging in voluntary negotiations. Their strategy hinges more on the skill levels of their workers than the leverage of strikes or official National Labor Relations Board elections, which use the state to compel the employer to negotiate. This unique approach often leads them to deviate politically from the broader US labor movement. Notably, they often oppose progressive environmental policies and tend to align more closely with the petrochemical industry on environmental issues. Additionally, they are not supportive of single-payer healthcare.

Some argue that this conservative stance of the Building Trades originates from their tradition of craft unionism, where workers are organized based on craft and skill rather than industry. However, historical-comparative analyses challenge this view. For instance, other unions with many members working in the petrochemical industry have backed progressive policies, including environmental policies, and other craft-based unions have endorsed single-payer healthcare despite organizing along craft lines instead of industry. The key distinction between these unions and their disparate political stances lies in the Building Trades' use of voluntary agreements, which minimizes conflicts with employers and constrains their ability to challenge management.

% Foner, Phillip S. History of the Labor Movement in the United States: Volume 2: From the Founding of the AFL to the Emergence of American Imperialism. New York: International Publishers, 1955; pp. 132–133.

\subsection{Craft Unionism vs. Industrial Unionism} \

Craft unionism is when workers are organized into a union by craft (occupation) rather than employer or industry. The earliest efforts in the US to organize workers collectively to assert the interests and preferences were generally organized around craft lines. Plumbers formed their union; machinists started one; so did the carpenters, and so on. Many of these early craft unions were successful, and eventually The American Federation of Labor (AFL) formed in 1886, bringing many of the labor unions at the time into a nationwide federation. Immediately many the constituent unions battled over craft jurisdiction. Who had the right to organize workers in the print room? Both the Machinists and the International Typographical Union claimed jurisdiction. As a result, one of the most important functions of the newly formed federation was to quell these jurisdictional fights.

Industrial unionism did not come until some time later. Industrial unionism is when workers are organized by employer or industry, regardless of their craft, occupation, or skill level. Many of the early craft unionists in the AFL were skeptical of indsutrial unionism. Many of the workers in mass production factories at the time were immigrants with no craft and little formal training. The craft union leaders in the AFL did not think that it would be possible to organize them because they did not have leverage that having a craft would equip the workers with. The jobs were largely repetitive and did not require a high-skill level. For example, standing in the same place and fastening the same bolts at the same place on a Model-T all day was not a craft in the way that being a machinist and working a lathe or being a pile driver and driving piles for a new bridge was.

\subsubsection{The Industrial Bargaining Approach}

The National Labor Relations Act (NLRA) sets forth the process that workers must follow to unionize a workplace. The National Labor Relations Board (NLRB) administers the NLRA. The NLRA specifies the processes that workers and union organizers should follow to form a union. The typical process begins with an effort to determine if there is a general interest in forming a union or joining a union among workers already hired. If there appears to be enough interest, “employee organizers typically collect union interest cards, petitions or other written statements from co-workers to show interest in union representation. Organizing efforts may be supported by an established union seeking to represent workers at a workplace. Workers may also form an independent union” (DOL n.d.:How can I form a union?). If enough cards are collected to demonstrate majority status, the workers may ask the employer to voluntarily recognize the union. If the employer refuses to voluntarily recognize the union, workers may strike and/or file a petition with the NLRB to request an election to certify the union and the collective bargaining representative of the workers (DOL n.d.; NLRB n.d.). Once the union has demonstrated that the majority of workers want union representation (either through voluntary recognition by the employer or through a NLRB election), the union and employer will attempt to reach an initial agreement. This type of agreement is called a Section 9(a) agreement.

\subsubsection{The Building Trades Approach}

In contrast to this approach, building and construction trade unions do not usually organize the workers at the workplace. They usually establish voluntary agreements with employers, which can be established before any employees are hired by the employer. Since these agreements can be established before any workers are hired, they are called prehire agreements, or in legalistic jargon, Section 8(f) agreements. Prehire agreements have a long history that predates their legitimation by Congress in 1959 with the passage of the Labor-Management Reporting and Disclosure Act (LMRA). In fact, prior to the LMRA, the NLRB had refused to take jurisdiction over the building and construction trades because their approach to union organizing was so different and in fact would be an unfair labor practice, a violation of the NLRA, if they were to take jurisdiction over the matter. So, whereas labor law is often discussed in terms of how the law constrains union organizing efforts, this is not the case with respect to prehire agreements used by the building and construction trades; they were already organizing this way prior to the passage of the LMRA.

This distinction between 9(a) and 8(f) agreements is useful because it underscores two very different orientations toward the boss and two very different approaches to mobilizing worker power. Perhaps most notable is the starting point of the unionization process. The former starts with the employees already hired. Sometimes it is primarily driven by union staff; other times, it can be driven more by shop floor organizers from the rank-and-file. Either way, the approach is to organize workers around issues they are facing in the workplace and compel the employer to negotiate a contract through coercive measures such as strikes or using the state as an instrument to compel the employer to bargain in good faith. The building and construction trade unions largely dispense with this approach, instead trying to entice employers on the basis of their members’ skill and training and by collaborating with the employers. (The building and construction trades are not forced by law to organize in this way; it is their choice whether to organize in this way or another way.)

\begin{figure}
  \centering
  \includegraphics[width=0.75\textwidth]{images/organizing_paths}
  \captionsetup{justification=centering, singlelinecheck=false, margin=2cm} % Adjust the margin as needed
  \caption{Shows both the industrial mode of organizing (left) and the construction mode of organizing (right). Construction unions may follow either path, but other unions may not voluntarily negotiate the way that construction unions can.}
  \label{fig:organizing_paths}
\end{figure}

% \subsubsection{Methods}

\newpage
\begin{center}
{\bfseries Notes}
\end{center}

\noindent
.
\newpage
\begin{center}
{\bfseries Bibliography}
\end{center}

% \printbibliography

\end{document}
