More recently, the geographer \poscite{browerbrownSolarFluxRemaking2023} dissertation focused on construction unions working on solar panel farms in the Central Valley. His principal focus was on how these unions "gained power"---secured jobs that might have otherwise gone to non-union employers. He contends that familial ties and ties with community groups where social reproduction occurs build the power necessary for these gains. While the empirical data from the project is fantastic, and the project offers an important insight into how communities organize alongside labor, it tends to paint a rosier picture than is warranted. Because it starts with the question of how the union \emph{gained power}, it tends to downplay the ways in which the unions' strategies were stuck within the fundamentally class collaborationist mode of organizing that has defined building trades organizing for decades. Admirably, \citeauthor{browerbrownSolarFluxRemaking2023}'s ethnographic and interview methods illuminate worker experiences in a research field where many are content not to investigate the activity of the rank-and-file. Lamentably, this has come at the cost of paying scant attention to what happens at the union hall.

Concern with the rank-and-file dominates the more left-inclined intellectual's research agenda, as the union leadership\footnote{Or what is often derisively called the "union bureaucracy" or "union officialdom."} is conceived of as reactionary, or at least more conservative than the rank-and-file membership. An emphasis on union democracy has, understandably, meant an emphasis on the dynamics in the workplace and, correspondingly, strengthened the focus on rank-and-file activity. Indeed, a common refrain from this perspective is that "[t]he workplace (not the union hall) is the starting point for union democracy" \parencite[1]{parkerDemocracyPowerRebuilding2005}. From this perspective, since employment starts at the workplace, that is where workers find their power, and the union hall is more or less tangential to the overall leverage and collective bargaining of the union. Thus, the researcher interested in how unions wield power ought to look at the job site and not the union hall.

This is a persuasive argument, and it would arguably apply to the building and construction trade unions if they organized like the rest of the labor movement. Alas, they do not. Unlike the rest of organized labor, construction workers are not hired and then organized; they are organized and then hired. Because of this, those who take the methodological approach described above overlook the opportunity to better understand how construction union-employer relations are actually forged, how and where those unions wield their power, and in which contexts that power is the strongest. This project seeks to illuminate these aspects and offer a clearer picture of how these relations are forged.
