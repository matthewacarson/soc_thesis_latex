Other authors explore the dynamics of race and gender in the building trades, highlighting the persistent hard-hat culture where masculinity is defined by physical prowess and bravery. They contend that this machismo is not only a reaction to the dangers of the job but also a way for workers to assert their identity in the face of unsafe working conditions and fear of reprisals \parencite{moccioContradictingMalePower1992, paapWorkingConstructionWhy2006}. Tradesmen often refuse to confront contractors about unsafe working conditions and instead reinforce a self-consciously macho image to cope with the risks without losing face \parencite{moccioContradictingMalePower1992}. \textcite{paapWorkingConstructionWhy2006} argues that the performative expectations of masculinity, which she terms "pigness," compel white male workers to engage in behaviors that ultimately weaken their position within the employment market. These behaviors include working harder and more dangerously to prove their worth to their employers, thereby commodifying their bodies in a way that erodes their negotiating power and long-term ability to work. \textcite{paapWorkingConstructionWhy2006} highlights that these practices not only harm individual workers but also undermine the union movement and the working class as a whole.

The image of the building trades as an Archie Bunkeresque "good old boys" club that excludes women and people of color is pervasive. However, it only offers an incomplete picture. For example, unions such as the \acrfull{liuna} are heavily Latino, but they are hardly bastions of egalitarianism. Indeed, sociologist David \textcite{fitzgeraldTransnationalismMexicanHometown2004}, in his study of transnationalism, found that the Southern California laborers local union in his study had a patronage network that excluded many first and second-generation Mexican-American laborers from the more desirable jobs in the union. Instead, \emph{Guadalupanos} were given priority based on their connections to hometown networks in Mexico. This hardly squares with the good old boys picture that one gets from \textcite{paapWorkingConstructionWhy2006}.

The emphasis on worker "machoness" also sidesteps the issue of employer responsibility. In a discussion of the "Big Blue" crane collapse that killed three workers at the then-under-construction Milwaukee Brewers' stadium, \textcite{paapWorkingConstructionWhy2006} contends that the workers' masculinities put them in harm's way. \citeauthor{paapWorkingConstructionWhy2006} contends that the men felt tremendous pressure to look tough and thus decided to conduct crane operations in high winds despite the danger and violation of \acrfull{oshact} regulations. \citeauthor{paapWorkingConstructionWhy2006}, like the employer, contends that the men voluntarily put themselves in harm's way instead of refusing to complete the unsafe task. However, this sidesteps the issue of holding the employer accountable and significantly overlooks the power differential between the employer and workers.

Workers, especially in construction work, have contingent, temporary employment and tend to have little protection from arbitrary firings. Though \emph{de jure} protection exists from retaliatory firings for refusing to do unsafe work both by law and typically also by \acrfull{cba} language, the contingent, temporary nature of the work makes it difficult to prove retaliation. Often, employers will retaliate with a "reduction in force" (i.e., a claim that there is a lack of work to keep someone employed) as a pretext. And since typically all that is needed to successfully fight a retaliation case is evidence that the employer had a legitimate business reason to lay someone off, the claim that the job was "winding down" will usually suffice absent some other evidence that shows that the employer harbored animus over a refusal to work in unsafe conditions.

Moreover, a reputation as a union "troublemaker" could likely follow one throughout their time within the union, even after a retaliatory "layoff." This, coupled with the fact that a layoff can often mean months of unemployment, makes refusing unsafe work much more difficult and fraught with worries about financial stability---paying rent, purchasing food, etc.---than \citeauthor{paapWorkingConstructionWhy2006} portrays it. So, while it is tempting to simply say that the workers should have turned down the unsafe task and rejected whatever masculine inclinations they have, the reality is much more complicated.

To be sure, there is significant occupational segregation in the building trades, mostly between unions classified as "basic crafts," laborers, roofers, cement finishers, etc., and the "skilled crafts," plumbers, pipe fitters, electricians, elevator constructors, etc. Blacks and especially Latinos are disproportionately employed in the "basic crafts," which usually have lower pay and less generous benefits, while the "skilled crafts" are disproportionately white.\footnote{Statistics by craft are difficult to find. Much of this is based on my observations during my ten years as a union construction worker. While I am fully aware that my experience might not be representative, it is well-documented that the "basic crafts" are comprised heavily of Latino immigrant workers (See, for example, \textcite{erlichStandingCrossroadsBuilding2005, grabelskyConstructionDeConstructionRoad2007}).} Across all unions, women are substantially underrepresented at only four percent of the unionized sector despite being 14 percent of the construction workforce \parencite{u.s.bureauoflaborstatisticsEmploymentWomenNonfarm2024, thewhitehouseReadoutMeetingNABTU2023}. So, while construction unions have made progress in terms of racial and ethnic representativeness, they have made much less progress in recruiting women.

But they are trying. \acrfull{nabtu} has commissioned diversity, equity, and inclusion reports to identify areas where the building trades can improve recruitment and retention of more diverse apprentice cohorts \parencite{bilginsoyDiversityEquityInclusion2023}. As laudable as this is, it likely will not improve the issues regarding workplace safety that \textcite{paapWorkingConstructionWhy2006} has highlighted. This is because the difficulty of challenging management stems from more complex sources, such as the way in which the bargaining relationship is forged with the employer, a topic that will be discussed at great length in the following sections.