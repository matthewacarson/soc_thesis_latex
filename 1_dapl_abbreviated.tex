% \section{An Outlier in the Labor Movement}\label{outlier}

% Unions often find themselves at the crossroads of political and social movements, particularly environmental movements, raising questions about their alliances and priorities. The \acrfull{uaw} could be negatively affected by transitions from fossil fuel-burning vehicles to electric insofar as those new plants may not be unionized, and it may result in job losses (\cite{feeleyElectricVehicleFactories2023}). The \acrfull{usw} faces more job losses at oil refineries\footnote{Some job losses were already brought on by the pandemic.} if climate policy restricts the operation of those facilities. Building Trades workers face unemployment if pipelines cease to be built and petrochemical refineries are shut down. Despite the contradictions, many labor unions have taken a long-term approach to the problem. They want green, environmentally-friendly jobs that are also good jobs with union wages and benefits. However, the Building Trades unions have not been among those unions.

\section{Dakota Access Pipeline and Standing Rock}\label{dapl}

The controversy surrounding the \acrfull{dapl} brought to the fore the political differences vis-\`a-vis the Building Trades' unions on the one hand and the rest of the labor movement on the other. The plan to build a 1,168-mile crude oil pipeline, called the \acrshort{dapl}, stretching from the Bakken and Three Forks production region of North Dakota to Patoka, Illinois, was announced in 2014 by Dakota Access, LLC (\cite{oconnellDakotaAccessPipeline2018, sahaFiveThingsKnow2016, usarmycorpsofengineersDakotaAccessPipeline}). The opposition to the pipeline culminated in protests in Sioux County, North Dakota, in 2016. The pipeline was slated to run through the Standing Rock Sioux Tribe reservation, which opponents of the project contended would “endanger[] sacred sites and drinking water” (\cite{sahaFiveThingsKnow2016}). The Standing Rock Sioux Tribe filed a lawsuit against the US Army Corps of Engineers in an attempt to halt the project. On September 9, 2016, a federal judge denied the tribe’s request to halt construction. However, within hours of the court’s ruling, the Obama Administration ordered the Army Corps of Engineers to put the project on hold and “determine whether it will need to reconsider any of its previous decisions regarding the Lake Oahe site under the \acrfull{nepa} or other federal laws” (\cite{officeofpublicaffairsJointStatementDepartment2016}). 

% Members of the Standing Rock tribe claimed that the project violated the principles of tribal sovereignty and threatened the continued existence of the tribe (\cite{massieUnderstandDakotaAccess2016}). Moreover, they claimed that the US government had not adequately consulted with the tribal government, which violated US treaties and the United Nations Declaration on the Rights of Indigenous Peoples (\cite{sahaFiveThingsKnow2016}). In addition to bringing thousands of Native American activists in opposition to the pipeline from across the country to Standing Rock, environmentalists, and climate activists were equally concerned not only with the environmental impact of expanding fossil fuel infrastructure but also the potential hazards of oil spills and other accidents endemic to the petrochemical industry (\cite{sahaFiveThingsKnow2016}). 


%\subsubsection*{Organized labor's response}

Many labor organizations supported the Native tribe and the environmentalists. For example, the \acrfull{seiu} emphasized that more than jobs were at stake; the pipeline would threaten the health and safety of “low-income communities and communities of color, including those where many \acrshort{seiu} members live and work” (\cite{nlfUnionsWeighDakota2016}). Other unions, including the National Nurses United and the Communication Workers of America, also expressed these solidarities with environmental and other movements (\cite{nlfUnionsWeighDakota2016}). Additionally, an entire coalition of trade unions and labor groups, the \acrfull{lcca}, collectively issued a statement opposing the pipeline. This coalition included the A. Phillip Randolph Institute, the Asian Pacific American Labor Alliance, the Coalition of Black Trade Unionists, the Coalition of Labor Union Women, the Labor Council for Latin American Advancement, and Pride at Work (\cite{apalaAFLCIOConstituencyGroups2016}).

% The group emphasized how labor’s interests are not at odds with environmental and Native American concerns and interests:

% \begin{quote}
% We remain committed to fighting the corporate interests that back this project and name this pipeline “a pipeline of corporate greed.” We challenge the labor movement to strategize on how to better engage and include Native people and other marginalized populations into the labor movement as a whole. (\cite{apalaAFLCIOConstituencyGroups2016})
% \end{quote}

% \noindent{}However, the \acrshort{lcca} also recognized that simply shutting down the construction of the pipeline would not be sufficient; the creation of good jobs to replace the ones lost by halting the pipeline’s construction was necessary:

% \begin{quote}
% Lastly, we applaud the many labor unions working to create a new economy with good green jobs and more sustainable employment opportunities for all. We also encourage key stakeholders — labor unions including the building trades, the Standing Rock Sioux tribe and others who would be impacted — to come together to discuss a collective resolution. (\cite{apalaAFLCIOConstituencyGroups2016})
% \end{quote}

% \noindent{}Lastly, the \acrshort{seiu} released a statement that emphasized the importance of both good jobs \emph{and} the environment. The statement is remarkably similar to the \textit{Just Transition} idea that the \acrfull{ocaw} first crafted decades earlier and that the \acrfull{usw} still embraces today:\footnote{The \acrshort{usw} is a union that both has many workers in the petrochemical industry \emph{and} supports green energy policy. They even support ceasing operations of many of the plants that their members work in as part of that commitment. This is discussed at greater length in \fullref{ocaw}.}

% \begin{quote}
% \acrshort{seiu} members recognize the importance of these jobs for these workers and their families and we demand that our government protect all workers whose lives and livelihoods are impacted by a shift away from fossil fuels. Our government must make the needed investments into building a new clean economy, including a just transition of workers from the fossil fuel workforce, by investing in clean energy and rebuilding and repairing much of our nation’s aging infrastructure, including existing pipelines which are in great need of repair. We will fight for an economy and democracy in which working families can live and work in a clean environment with good jobs for all. (\cite{nlfUnionsWeighDakota2016})
% \end{quote}

% The \acrshort{seiu} and other unions opposed to the pipeline were engaged in a collective struggle against the company that planned to build a pipeline through the community, polluting the land and water without any respect to indigenous groups, such as the Standing Rock Sioux. The \acrshort{lcca} made clear in its statement that it was against the "corporate greed" that the project represented (\cite{apalaAFLCIOConstituencyGroups2016}).

%\subsubsection*{The Building Trade's response}

\glsentrylong{nabtu}, a trade department of the \acrshort{aflcio} representing 14 constituent building trades unions, issued a statement within days condemning the actions of the Obama Administration. On September 14, 2016, they expressed their “disappointment” with the Obama Administration in the latter’s willingness to “halt the lawful construction” of the \acrshort{dapl}, emphasized the advanced training of their skilled craftworkers, and contended that many safety redundancies would be in place while the project was underway  (\cite{nabtuStatementObamaAdministration2016}). Further, they chided the Executive Branch for its “disregard” of “the facts,” the “exhaustive permitting and review process, stakeholder engagement,” and “the ruling of a Federal Court Judge,” and they lamented the loss of construction jobs with “family-sustaining wages and benefits” (\cite{nabtuStatementObamaAdministration2016}).

% Early the following month, the \acrshort{nabtu}, with signatures from the presidents of five construction unions, wrote a letter directly to the White House reiterating their concerns. In the letter, they underscored that many of their 8,000 members who were working on the project were already struggling and facing hardship because of unemployment caused by the Executive Branch’s intervention. Additionally, they emphasized that this intervention sets a “chilling” precedent that would undermine “future investment in necessary U.S. infrastructure---from highways and bridges to ports and factories,” and that “If companies like Energy Transfer Partners cannot trust that the regulatory process outlined in federal law will be upheld, who will continue to invest in America?” (\cite{callahanLetterObama2016}).

\section{Union Members as Shareholders}\label{union_shareholders}

The \acrshort{nabtu} has clearly taken a very different position than the rest of organized labor. But why are the building trades taking such a different stance? One might think that they also would be guided by a broader notion of unionism; after all, their members also live and work in the communities that are being harmed by global warming. What principles guide this sort of unionism?

A 2015 interview with Sean McGarvey, President of \acrshort{nabtu}, provides some answers. McGarvey met with Martin Durbin, President and CEO of \acrfull{anga}, for an interview later released on YouTube. When Durbin asked McGarvey how he thought the relationship with management had been progressing, McGarvey responded:

\begin{quote}
It’s really been interesting to work on building these relationships. We have so much in common as opposed to the things we disagree with, and we've gotten together and said, “Let's really examine these things we have in common,” and then once we examine them, we said, “Well, how do we partner to move the issues along that we really agree with that are in the best interest of our country and our economy?” And [we’ve had] the opportunity to work with really smart people in really iconic great companies that have been around for a hundred years or have been around for 20 years\ldots{}I serve a membership; my job is to create economic opportunity for my membership\ldots{}That's what I'm supposed to do and when you run a company, you have a board of directors, and you have shareholders to answer to if you're publicly traded. Both of those things are true, but then there's this meeting in the middle: how do we create value for shareholders? How do I create value for my members and do it in a responsible way, working with partners who want to work with us responsibly to create value for their shareholders?\ldots{}[W]hen you have the opportunity to have those conversations you say, “gosh there [are] so many things that we can do together and do better together,” and I [have] got to tell you\ldots{}it's always kind of been the way [of] the building trades. If you go back to a great quote from George Meany, back when he was a plumber in New York City and a local leader, they never went on strike; they never had a strike, even during that tumultuous time down there. It's because [the union’s] contractors needed to be successful for his plumbers to work. (\cite{natgasnowNextInfrastructureChallenge2015})
\end{quote}

\sloppy
McGarvey’s answers exemplify the class collaborationist approach taken by the \acrshort{nabtu}. Contra other unions, the \acrshort{nabtu} has prioritized its relationship with industry over solidarities with the Native American tribes and environmental groups. McGarvey made very clear that he thinks the building trades should prioritize their relationships with management by finding the things that they have in common over policies that might benefit workers or the community more broadly. The analogy of a building trades leader to essentially a CEO is equally telling. From this perspective, the union members become not much more than “shareholders,” who pay dues and other fees in exchange for a “return” in the form of employment covered by a collective bargaining agreement. Rather than these agreements being forged as a result of workplace organizing, where work stoppages or other such actions are the crux of such struggles, they are forged in labor-management board meetings, where the parties find ways to collectively create “value” for their shareholders or members. Indeed, as the George Meany reference makes clear, strikes are rare to nonexistent by design.
\sloppypar

% Of course, strikes are rare in general, but as a relative measure, they are even less frequent in construction. In 2022, 14.3 million workers were unionized; about 3 million (20.9 percent) of them work in construction (\cite{blsUnionMembersSummary2024}). That same year, there were 413 labor actions, such as strikes or other work stoppages, but only three were in the construction industry (0.73 percent), meaning that 99.3 percent of labor actions were outside of the construction industry (\cite{ilrschoolILRLaborAction}). This is about one-tenth the “expected rate” if construction workers had labor actions at a rate equal to their proportion of the unionized workforce. That is, strike activity, as rare as it might be, is coming disproportionately from non-construction sectors (Table \ref{tab:strikes}).

% \begin{table}[h]
\centering
\begin{tabular}{llll}
                    & All            & \multicolumn{2}{l}{Construction} \\ \hline
Unionized Workforce & 14.3 million	& 3 million				& 20.9 \%     \\
Labor Actions       		& 413				& 3							& 0.73 \%    
\end{tabular}
\captionsetup{justification=centering, singlelinecheck=false, margin=2cm} 
\caption[Labor Actions]{Construction unions represent nearly 21 percent of unionized workers but were involved in less than one percent of labor actions in 2023 (\cite{ilrschoolILRLaborAction}).}
\label{tab:strikes}
\end{table}

% It is also true that class collaboration is not limited to petrochemical work and that the building trades could, at least in theory, choose class collaboration with the green energy industry. But this would only be possible if the green energy industry had a reason to collaborate in the first place. Recall that construction collective bargaining agreements are voluntary and that construction employers usually only have an incentive to sign these agreements if their jobs require highly skilled labor or if they would benefit from the quick deployment of a large number of workers. Green energy companies have, by and large, been able to build without union labor (\cite{scheiberBuildingSolarFarms2021}), suggesting that they either do not need these things or have been able to obtain them elsewhere (which would mean that the union is losing its monopoly of these commodities). So, there is no reason for those employers to pay union wages and into union benefits and pension funds if they don’t need to. Simply put, green energy has largely not needed construction unions, and thus, construction unions have largely failed to make inroads in green energy.
