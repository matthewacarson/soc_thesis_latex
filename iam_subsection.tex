The \acrfull{iam} and \acrshort{ua} are both craft unions. Both were \acrfull{afl} unions before the \acrshort{afl} and \acrshort{cio} merged in 1955. Although perhaps not anchored as steadfastly to the expansion of social-wage policy and decommodified public goods as unions such as the National Nurses United and National Health Care Workers or even The Brotherhood of Maintenance of Way Employees, they have nonetheless given their official support toward the effort to establish single-payer healthcare in the United States, a policy that would eliminate private insurers and establish government-funded healthcare insurance that would cover all regardless of one's income or finances. This arguably is along the more "radical" edge of the US trade union movement, such as it is. What could explain this different outcome? Both the \acrshort{iam} and the building trades unions were early \acrshort{afl} craft unions. They both have a history of racist, exclusionary policies. A key difference, however, is that the \acrshort{iam} is not a construction union and thus cannot enter into voluntary "prehire" agreements; they cannot "organize the bosses."

Boeing workers in Washington's Puget Sound area formed \acrshort{iam} District Lodge 751 in 1935. This places the formation of the union directly in the context of the years in which the fledgling \acrshort{cio} was organizing workers in the rapidly expanding industries of the time. The \acrshort{cio}'s organizing tactics were influenced by the \acrfull{iww}, a radical union that pushed for the replacement of wage labor and the private ownership of industry with industrial democracy, where workers would collectively own the means of production and collectively make decisions about how the workplace should be run (\cite{industrialworkersoftheworldIWW}). Though the \acrshort{iww} remained politically marginal, their tactics and philosophy undoubtedly had an impact on the \acrshort{cio} in the 1920s and 1930s (\cite[2-6]{mccannBloodWaterHistory1989}). The \acrshort{afl}, committed to craft unionism, was unable to organize the new industrial workers because of disputes that would arise regarding which craft union the newly organized industrial workers should join. When the AFL did organize industrial workers, they would first organize them into industrial-like union structures called "federal" unions. Eventually, these unions were supposed to dissolve, with the workers reassigned to the appropriate craft union. However, many of the newly organized industrial workers were not attracted to the craft union idea, as evidenced by their continuing attendance at the "Federal" (i.e., industrial) union that they were initially organized into and their failure to attend the union meeting for their particular craft (\cite[8]{mccannBloodWaterHistory1989}).

In the 1930s, when the Boeing shop first unionized, the \acrshort{cio} was primarily focused on the auto industry. As a result, there was not much emphasis from the CIO on organizing in the aerospace industry at that time. The workers experienced terrible conditions at work, and there had been talk about forming a union. Boeing had considered forming a company union, a union that is controlled by the employer and is not a truly independent labor organization, which was a popular tactic for many companies at the time. But with the passage of the \acrshort{nlra}, company unions became illegal, and Boeing recognized that, despite the pending legal challenges against the \acrshort{nlra}, the prospect of successfully forming a company union seemed unlikely; indeed, shortly after the company announced its intent to create a company union, the District Lodge filed a \acrshort{nlrb} complaint, and Boeing retracted its plan (\cite[23-24]{mccannBloodWaterHistory1989}). Furthermore, the \acrshort{cio} was on the rise and mounting militant challenges to employers in the auto industry, and Boeing thought that it would only be a matter of time before the growing labor federation turned its sights on Boeing. District Lodge 751, by contrast, was rather small, with only 35 members, and was not anchored in radical unionism \`{a} la the \acrshort{iww} or \acrshort{cio}. (\cite[24]{mccannBloodWaterHistory1989}). Within this context, it made quite a lot of sense for Boeing to choose what it saw as a lesser evil. So it chose to recognize District Lodge 751 as the workers' union.

The initial years of bargaining were generally amicable. Boeing agreed to all of 751's "no-cost" demands so long as 751 was willing to forgo any wage increases or other items that would directly increase Boeing's labor costs and alarm their creditors \parencite[96--97]{mccannBloodWaterHistory1989}. But the friendly relations between District Lodge 751 and Boeing were short-lived. During the war years, the company grew, and along with it came the professionalization of negotiations within Boeing. After the death of Boeing President Phil Johnson in 1945, with whom the union had such great relations that they commemorated him in the union magazine as "a great guy," the new Boeing President, William M. Allen, dispensed with the friendly relations and began to target the seniority system in an effort to allow for complete freedom of transfer rights within the plant \parencite[98--99]{mccannBloodWaterHistory1989}. Allen immediately began layoffs at the plant, and seniority-related rehiring procedures became contentious as the company rehired supervisors first irrespective of seniority \parencite[100]{mccannBloodWaterHistory1989}. 

The union fought Boeing's violation of the contract and prevailed in arbitration, but Boeing now had its sights set on eliminating the seniority provisions altogether. In 1947, negotiations stalled as Boeing insisted on provisions that "would have effectively negated the seniority protections gained by the Union\ldots{}" \parencite[101]{mccannBloodWaterHistory1989}. Negotiations stalled, and the union called a strike but were hobbled by the no-strike clause in their agreement and its indefinite duration; because of this, the International refused to sanction the strike \parencite[105--6]{mccannBloodWaterHistory1989}. In spite of this, the union ultimately struck and lost; in the end, they returned to work without any union contract at all \parencite[107--8]{mccannBloodWaterHistory1989}. Most significantly, amicable relations between 751 and Boeing were over.

Not only had District Lodge 751 lost the strike, but since they had a no-strike clause in their contract, the strike was deemed illegal by the courts, and the union lost its designation as collective bargaining agent at Boeing. To make matters worse, the Teamsters Union, aided by the \acrfull{afl} and the local building trades unions, was trying to raid the shop, meaning the Teamsters sought to gain control as the sole union at Boeing \parencite[131--39]{mccannBloodWaterHistory1989}. A new \acrshort{nlrb} election was held, and ultimately, 751 won and regained its position as the Boeing workers union \parencite[139]{mccannBloodWaterHistory1989}. However, it was now in a much weaker position as the union security clause\footnote{A union security clause is a clause in a collective bargaining agreement (CBA) between an employer, a union, and a bargaining unit that requires employees to join the union as a condition of employment.} and seniority provisions were now a thing of the past.

Since then, District Lodge 751 has had to fight to regain (and retain) any of the provisions that it had during World War II. Negotiations in 1962 nearly led to another strike when Boeing again demanded major concessions from the union, and in 1965, the union struck to eliminate the "Performance Analysis System" that Boeing had unilaterally instituted after the defeat of the seniority system in 1948 \parencite[174]{mccannBloodWaterHistory1989}. Since then, the union has had to expand its organizing to include more than just Boeing employees. The union recognized the importance of such organizing efforts as the production process has become more fragmented and outsourced, and not all parts are produced in-house now \parencite[178--83]{mccannBloodWaterHistory1989}. From that perspective, organizing by craft the way that the building trades does is not a viable option. As a result, the union has had to continue to pursue more industrial-style organizing, which is an inherently more conflictual strategy despite being a craft union. 

However, that does not mean that the \acrshort{iam} has completely abandoned its craft union roots. Elsewhere on the West Coast, the \acrshort{iam} still is involved in "union turf wars" that are characteristic of craft unions. For example, the \acrshort{iam} and \acrfull{ilwu} have battled over jurisdiction at West Coast ports in the past decade, with the \acrshort{iam} contending that only the skilled craftworkers in the Machinists are qualified to perform the diagnostic and maintenance work at the port, while the \acrshort{ilwu}, which has traditionally been an industrial or "wall-to-wall union," insists on its jurisdiction over all workers at the ports irrespective of their title or job classification \parencite{mongelluzzoUnionTurfWars2014}. As the ports modernize, they are expected to bring in automated stacking cranes that are "computerized and highly technical" \parencite{mongelluzzoUnionTurfWars2014}. Don Crosatto, \acrshort{iam} senior area director, argues that "You can’t just hand a longshoreman a wrench and expect him to be a mechanic" and that only trained craftworkers who have completed an apprenticeship are equipped to do the demanding, technical work; "We are a craft union. We’re not embarrassed by that," he added \parencite{mongelluzzoUnionTurfWars2014}.

Notwithstanding the craft union orientation of the \acrfull{iam}, the union has taken the bold step of endorsing single-payer healthcare. They have affiliated with the \acrfull{lcsp} and made an official commitment by passing a resolution at the union's 2004 convention \parencite{unionsforsinglepayerhealthcareUnionsSinglePayer}. No construction unions are affiliated\footnote{Affiliate is a higher standard than merely expressing support. It requires the union to pass an official resolution and a commitment to a single-payer healthcare worker education program.} with the \acrshort{lcsp}, and none that this author is aware of have passed official resolutions expressing support for the same, though there have been sporadic expressions of support for single-payer from the \acrlong{ibew}\footnote{Author correspondence with Mark Dudzic (\acrshort{lcsp}).} and the \acrlong{iupat} \parencite{unionsforsinglepayerhealthcareUnionsSinglePayer}. So while the \acrshort{iam} has not shed its craft union roots entirely, it embraces a more class struggle-based unionism by endorsing policies that would benefit not just its members but the working class more generally.