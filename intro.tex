Do the ways that unions organize affect their political stances? US construction unions have a distinct approach to organizing their workers vis-à-vis other labor unions. While most labor unions typically compel employers to negotiate through secret ballot elections or work stoppages, the Building Trades take a different route by engaging in voluntary negotiations.\footnote{See \autoref{modes_of_bargaining} for a discussion of the history of secret ballot elections and how the Building Trades negotiates.} Their strategy hinges more on the skill levels of their workers than the leverage of strikes or official \acrfull{nlrb} elections, which use the state to compel the employer to negotiate. At the same time, the Building Trades are often outliers in the US labor movement. They frequently oppose progressive environmental policies, and thus align more closely with their employers in the petrochemical industry than with other labor or environmental organizations on environmental issues. Additionally, they are not supportive of single-payer healthcare or other social wage policies.\footnote{To be sure, not all non-Building Trades unions endorse or support single-payer healthcare, but \emph{no} Building Trades unions do at the national level.} Why have the Building Trades taken these positions?

One potential explanation is that unions with many workers in the petrochemical industry will align with the employer on environmental policy if they believe environmental policies might be "job killers." That is, the Building Trades' opposition to an environmental policy meant to attenuate global warming might simply be a function of the union's interest in keeping their members working. The building trades have many members working on projects in the petrochemical industry, and from this perspective, this is what one would expect from workers and their unions in that industry. After all, why would they support something that might cause unemployment?

Still, others argue that the conservative stance of the Building Trades originates from their tradition of craft unionism, where workers are organized based on craft and skill rather than industry (\cite{roginComment1974, perlmanHistoryTradeUnionism1922, issermanGodBlessOur1976, fonerHistoryLaborMovement1996}). Craft unions were some of the earliest labor organizations in the US. Their focus on organizing narrowly based on craft distinctions (e.g., plumber) rather than by industry (e.g., construction worker) often corresponded with nativist and racist policies and more conservative positions (e.g., anti-communism and redbaiting), while industrial unions more often opposed racism and were more militant (\cite{fonerHistoryLaborMovement1994}). In short, craft unions have tended to look out for "their own" more than workers more broadly.\footnote{There are dissenting views concerning how politically conservative the \acrshort{afl} craft unions were. See \citeauthor{cobblePureSimpleRadicalism2013} (\citeyear{cobblePureSimpleRadicalism2013}) for a dissenting scholarly article, and \citeauthor{parkerAreIndustrialUnions2008} (\citeyear{parkerAreIndustrialUnions2008}) for a union activist's perspective on the nature of craft and industrial unionism.} The Building Trades unions continue to operate as craft unions today.\footnote{One can find a union for almost every construction craft---electricians, plumbers, bricklayers, and so on---but good luck finding a union called the "Construction Workers Union."} From this perspective, the Building Trades' conservativism stems from their narrow, craft-based unionism.

However, these views offer an incomplete picture. For instance, some unions with many members working in the petrochemical industry, such as the United Steelworkers (\acrshort{usw}), have backed progressive policies, including environmental policies, and other craft-based unions, such as the \acrfull{iam}, have endorsed single-payer healthcare despite organizing along craft lines instead of industry. The key distinction between these unions and their disparate political stances lies in the Building Trades' use of voluntary agreements, which minimizes conflicts with employers and constrains their ability to challenge management.
