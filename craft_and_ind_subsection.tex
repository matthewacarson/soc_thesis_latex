Craft unionism is when workers are organized into a union by craft (occupation) rather than by industry or employer (\cite[97]{suffernCraftVsIndustrial1936}).\footnote{For a discussion on the differences between industrial and craft unions from a union activist perspective, see Parker (\citeyear{parkerAreIndustrialUnions2008}).} \citeauthor{ohCraftIndustrialUnions1989} has defined craft unions and industrial unions concisely:

\begin{quote}
Craft unions are those whose jurisdiction concerns a particular skilled occupation, such as carpenters, plumbers, and painters, in which membership is a result of employing a particular occupation, regardless of employing industry. On the other hand, industrial unions define their jurisdictions in terms of particular industries, such as autoworkers and steelworkers. (\citeyear[2]{ohCraftIndustrialUnions1989})
\end{quote}

\noindent{}William Green, former secretary-treasurer of the \acrfull{umwa}, similarly defined industrial unionism as "the organization of all men employed in an industry into one compact union," while "Craft unionism means the organization of men employed in their respective crafts resulting in numerous organizations within a particular industry" (\cite[69–70]{sapossIndustrialUnionism1935}).

The earliest efforts in the US to organize workers collectively to assert their interests and preferences were generally organized around craft lines. Plumbers formed their union; machinists started one; so did the carpenters, and so on. Eventually, the \acrfull{afl} formed in 1886, bringing many of these labor unions into a nationwide federation (\cite[97]{suffernCraftVsIndustrial1936}). Almost immediately, constituent unions battled over craft jurisdiction. For example, who should operate the mobile concrete mixers on the job site? Both the Teamsters and Operating Engineers claimed the work to be theirs at one point, leading to a jurisdictional dispute (\cite{jaffe1940}). This is a feature of craft unionism that makes it more conservative: it pits workers and their unions against one another. Therefore, one of the most important functions of the newly formed federation was to quell these jurisdictional fights (\cite{jaffe1940}).

Industrial unionism did not emerge until sometime later. Industrial unions organize workers by employer or industry, irrespective of their craft, occupation, or skill level. Many early craft unionists in the \acrshort{afl} were skeptical of industrial unionism; they did not think that it would be successful because many workers in mass-production factories were immigrants with no specific craft or formal training and thus lacked the leverage typically associated with having a craft (\cite{fonerHistoryLaborMovement1994, fonerHistoryLaborMovement1981a}). In contrast to a craft, like a machinist or a carpenter, which requires the worker to know all parts of the production process from beginning to end, industrial work is usually repetitive and requires little skill or knowledge of the entire production process. Because of this, craft unionism left industrial workers out of organized labor, and industrial unionists saw craft unionism as sclerotic.

Industrial unionism was largely a response to these perceived limitations of craft unionism. Exponents of industrial unionism believed that these limitations could be overcome by adopting the industrial unionist approach---organizing an entire industry regardless of occupational or craft classification. Sometimes this is called "wall-to-wall" unionism (\cite{meyersWhatAreMy2023}). While craft unionism is largely associated with parochialism and thus a more conservative outlook, industrial unionism is often associated with a more progressive or even radical unionism.\footnote{Labor historians have debated the extent to which the \acrshort{afl}'s unionism was "declensionist"---conservative, "disengaged," "voluntarist," lacking class consciousness and an "agenda\ldots{}of change for the nation as a whole" (\cite[61–62]{cobblePureSimpleRadicalism2013}). I emphatically do \emph{not} intend to take a position on this debate. Instead, I have tried to present, as best I can, the basic outline of the perspective that depicts the \acrshort{afl} unions as conservative without adopting this view as "correct."}