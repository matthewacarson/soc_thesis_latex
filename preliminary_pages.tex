%%%% Title Page %%%%
\begin{document}
\setstretch{1.25} % Set line spacing
\begin{titlepage}
  \thispagestyle{fancy}
  \pagenumbering{gobble} % Turn off page numbering for the title page
  \fancyhead[L]{Running Head = \shortTitle}
  \renewcommand{\headrulewidth}{0pt} % header rule
  \centering
  \vspace*{2in}
  \fullTitle\par
  \vspace{1.2in}
  {Matthew A. Carson\par}
  \vspace{12pt}
  Department of Sociology\par
  University of California, Los Angeles\par
  \vspace{0.5in}
  {\today\par}
%  \vfill
%  \wordcount
\end{titlepage}

% Blank page so that the abstract does not begin on the back of the cover page.
% Blank page
\thispagestyle{empty} % Remove header and footer

\vspace*{\fill}
\hspace*{\fill}
\begin{center}
    \noindent{}\textit{This page intentionally left blank}
\end{center}
\hspace*{\fill}
\vspace*{\fill}

\clearpage

% Set page numbering to Roman for preliminary pages
\pagenumbering{roman}
\setstretch{2}
%\titlespacing*{\abstract}{0pt}{0pt}{0pt}

% Add abstract page
\begin{abstract}
US Building Trade unions organize their workers differently. Most labor unions compel employers to negotiate, but the Building Trades engage in voluntary negotiations, relying on workers' skill levels rather than strike leverage. This approach correlates with their frequent political deviations from the broader US labor movement, particularly in opposing progressive environmental policies, aligning more closely with the petrochemical industry on environmental issues, and not supporting single-payer healthcare. One view is that unions pursue their members’ interests narrowly, sacrificing broader working-class interests if they feel it is necessary to secure work for their members, and some suggest that the conservative stance of the Building Trades stems from their craft union tradition, in which workers are organized by craft and skill instead of by industry. However, using historical-comparative methods, I show that these arguments do not hold. Petrochemical unions have supported progressive policies, and other craft-based unions have endorsed single-payer healthcare. However, unlike the Building Trades, those unions have never used voluntary agreements. Consequently, they have experienced more conflicts with employers. These findings challenge traditional views and suggest that the Building Trades' conservative negotiation strategies significantly shape their political and policy positions, reinforcing an employer-union dynamic that limits challenging management.
\end{abstract}

\clearpage

% Add table of contents (TOC)
\setstretch{1.25}
\hypertarget{toc}{}
\tableofcontents
%\clearpage

% List of Figures
\listoffigures

% List of tables
\listoftables

\clearpage
